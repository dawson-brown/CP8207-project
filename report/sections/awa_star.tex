\section{Background}
For this study, I drew on \awa as well as concepts in knowledge free exploration techniques.

\subsection{AWA*}
Hansen and Zhou introduced a method for converting Weighted A* into an Anytime algorithm giving \awa \cite{hansen2007anytime}. To do so, they noted three things that are required:

\begin{enumerate}
    \item An inadmissible evaluation function for selecting nodes for expansion. In the case of \awa, that's a weighted, admissible heuristic \cite{hansen2007anytime}.
    \item The search must continue after a (likely suboptimal) solution is found so that improved solutions can be found \cite{hansen2007anytime}.
    \item An admissible evaluation function which serves as a lower bound on the optimal solution, along with the upper bound being the cost of the best solution found so for; these bounds can be used to prune the open list until it's empty at which point the last found solution must be optimal \cite{hansen2007anytime}.
\end{enumerate}

Algorithm \ref{alg:awa} gives the details of the `anytime' part of \awa--the steps omitted at line 14 are those that are part of normal WA$^*$.

\begin{algorithm}
\caption{AWA*}\label{alg:awa}
\begin{algorithmic}[1]
\Require $f'(n)=g(n)+wh(n)$, $f(n)=g(n)+h(n)$
\State $g(init) = 0$
\State $OPEN \gets \{init\}, CLOSED \gets \emptyset$
\State $best \gets $ None, $f(best) \gets \infty$ 
\While {$OPEN \neq \emptyset$}
    \State $n \gets \argmin_{n' \in OPEN}f'(n')$
    \If {$f(n) < f(best)$}
        \State $CLOSED \gets CLOSED \cup \{n\}$ 
        \For{$n_i \in Succ(n)$ \textbf{if} $f(n)<f(best)$}
            \If{$n_i$ is goal}
                \State $f(n_i) \gets g(n_i)+c(n,n_i)$ 
                \State $best \gets n_i$  
                \State Continue
            \EndIf
            \State \vdots
        \EndFor
    \EndIf
\EndWhile
\State \Return $best$
\end{algorithmic}
\end{algorithm}

This approach is largely agnostic to the internals of the search being done so long as nodes are explored in some best first ordering that allows for open list pruning \cite{hansen2007anytime}. This makes it possible to include randomized exploration techniques in \awa.

\subsection{Randomized Exploration}
Knowledge-free, randomized exploration has shown promise in satisficing search algorithms, in particular GBFS \cite{valenzano2014comparison}. Here, Valenzano \textit{et al.} describe $\epsilon$-GBFS in which, with probability $(1-\epsilon)$, the node with the smallest heuristic value is chosen for expansion, otherwise, a node is chosen uniformly at random from the open list \cite{valenzano2014comparison}. In their experimentation, adding $\epsilon$-greedy node selection showed considerable improvement in several domains \cite{valenzano2014comparison}.

Despite showing improvement in some domains, and its relative simplicity, strictly random exploration (with probability $\epsilon$) runs into problems when large local minima or plateaus are encountered. This is because the open list will come to be dominated by that plateau and so sampling randomly will likely expand a node on the plateau which is in part what exploration is trying to avoid \cite{xie2014type}\cite{cohen2021type}. To address this, Xie \textit{et al.} proposed an exploration technique based on a type system giving Type-based exploration in which nodes are bucketed by g-cost and their heuristic value. During exploration, a bucket is first sampled uniformly at random and then a node within that bucket is sampled \cite{xie2014type}. This approach achieves a much better spread over the state space than simple \egreedy sampling \cite{xie2014type}.

Cohen \textit{et al.} showed that WA$^*$ can also fall prey to these large plateaus. To alleviate this, they developed a modified Type-based exploration technique that maintains the bounded suboptimal guarantees of WA$^*$ giving Type-WA$^*$ \cite{cohen2021type}. 

