\section{Introduction}
Algorithms like A$^*$ which are guaranteed to return the optimal solution to a search problem are often untenable in large problem spaces due to the time and space requirements. To address this, bounded suboptimal search algorithms, like Weighted A$^*$ (WA$^*$) and satisficing search algorithms like GBFS, can run faster with lower resource consumption in exchange for returning a suboptimal solution. WA$^*$ works by weighting an admissible heuristic to create an inadmissible one, Hansen and Zhou took advantage of this to create Anytime WA$^*$ in which a solution found with an inadmissible heuristic can be slowly refined overtime with the admissible one giving a sequence of ever better solutions; in this way they achieve both desired outcomes--fast optimal solutions, and in the time the optimal solution \cite{hansen2007anytime}.  

Suboptimal Search algorithms have been shown to benefit a great deal from knowledge-free randomized exploration. GBFS has been shown to improve from both random exploration and from exploring according to a type system \cite{valenzano2014comparison}\cite{xie2014type}. Type-based exploration has also been extended to work in WA$^*$ \cite{cohen2021type}.

In this project, I wanted to examine anytime search algorithms that made use of various exploration techniques over the course of their search for the optimal solution. To start, I paired \awa with \egreedy exploration, the simplest exploration technique. I then created my own, very simple exploration technique termed \ebgreedy in which nodes are sampled for expansion non-uniformly according to a Beta distribution $B$. I then paired \ebgreedy exploration with \awa. Taking standard \awa as a baseline, I evaluated the three algorithms to see if exploration could be beneficial in anytime heuristic search. The main contributions of this project are the following:
\begin{enumerate}
    \item I combine \egreedy exploration with \awa to give the algorithm \eawa.
    \item I introduce \ebgreedy exploration and combine it with \awa to give \ebawa.
    \item I do an in depth comparison of the three algorithms, \awa, \eawa, and \ebawa, on the unit cost sliding tile puzzle using the Manhattan Distance Heuristic; I briefly look at the inverse cost sliding tile puzzle; and lastly I touch on the exploration techniques alongside the Correct Tile Placement Heuristic. 
\end{enumerate}

The rest of this paper is structured as followed: first I go over the relevant background including \awa and various exploration techniques from the literature; I then describe in detail the two algorithms \eawa and \ebawa; this is followed by the evaluation of three algorithms; lastly there's a brief discussion and conclusion.

