\section{Introduction}
Algorithms like A$^*$ which are guaranteed to return the optimal solution to a search problem are often untenable in large problem spaces due to the time and resource requirements. To address this, bounded suboptimal search algorithms, like Weighted A$^*$ (WA$^*$) and satisficing search algorithms like GBFS, can run faster with lower resource consumption in exchange for returning suboptimal solutions. WA$^*$ works by weighting an admissible heuristic to create an inadmissible one, Hansen and Zhou took advantage of this to create Anytime WA$^*$ (\awa) in which a solution found with the weighted, inadmissible heuristic can be slowly refined overtime with the admissible heuristic giving a sequence of ever better solutions; in this way they achieve both desired outcomes--fast suboptimal solutions, and, in time, the optimal solution \cite{hansen2007anytime}.  

Suboptimal Search algorithms have been shown to benefit a great deal from knowledge-free randomized exploration. GBFS has been shown to improve from both random exploration and from exploring according to a type system \cite{valenzano2014comparison}\cite{xie2014type}. Type-based exploration has also been extended to work in WA$^*$ \cite{cohen2021type}. Suboptimal search algorithms that employ some form of randomized exploration are operating under the assumption that the heuristic and evaluation function used can be led astray in many domains falling into large local minima that are not a part of simpler solutions that the evaluation function can't see. Exploration can mitigate this by devoting time to exploring outside of the states that would normally be expanded.

In this project, I wanted to examine anytime search algorithms that make use of various exploration techniques over the course of their search for the optimal solution. To start, I paired \awa with \egreedy exploration, the simplest exploration technique. I then created my own, very simple, exploration technique termed \ebgreedy in which nodes are sampled for expansion non-uniformly according to a Beta distribution $B$. I then also paired \ebgreedy exploration with \awa. Taking standard \awa as a baseline, I evaluated the three algorithms to see if exploration could be beneficial in anytime heuristic search. The main contributions of this project are the following:
\begin{enumerate}
    \item I combine \egreedy exploration with \awa to give the algorithm \eawa.
    \item I introduce \ebgreedy exploration and combine it with \awa to give \ebawa.
    \item I do an in depth comparison of the three algorithms, \awa, \eawa, and \ebawa, on the unit cost sliding tile puzzle using the Manhattan Distance Heuristic; I briefly look at the inverse cost sliding tile puzzle; and lastly I touch on the Correct Tile Placement Heuristic to see if exploration can help when a very uninformed heuristic is used. 
\end{enumerate}

The rest of this paper has the following structure: first I go over the relevant background including \awa and various exploration techniques from the literature; I then describe in detail the two algorithms \eawa and \ebawa; this is followed by the evaluation of three algorithms; lastly there's a brief discussion and conclusion.

